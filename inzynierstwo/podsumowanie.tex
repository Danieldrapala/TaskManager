\chapter{Podsumowanie}

\section{Podsumowanie pracy}
\thispagestyle{chapterBeginStyle}

Celem niniejszej pracy inżynierskiej było stworzenie aplikacji internetowej wspomagającej zarządzanie zadaniami, w oparciu zastosowaniu metody Kanban.  Głównym zadaniem aplikacji było usprawnienie pracy własnej lub pracy w grupie, przy odpowiednim podziale zadań oraz ustalaniu priorytetów. Co ciekawe, aplikacja pozwala zorganizować codzienne obowiązki i efektywnie wykorzystać czas, ale również jest przydatna w przedsiębiorstwach przy organizowaniu i nadzorowaniu pracy zespołu. Zaproponowana aplikacja pozwala na rezygnację z notatek, wszystkie plany i zadania możemy zapisać w jednym miejscu, modyfikować w dowolnym czasie, potrzebujemy dostępu do urządzenia elektronicznego i internetu. Popularność istniejących na rynku aplikacji do planowania dowodzi, że istnieje spora grupa użytkowników, która chętnie skorzysta z zaproponowanego rozwiązania. 

\section{Plany rozwoju aplikacji}
Po zapoznaniu się z opiniami pierwszych użytkowników ważnym kierunkiem rozwoju aplikacji jest poprawa interfejsu użytkownika. Po okresie próbnym korzystania z aplikacji przez użytkowników, należy przeprowadzić ankietę lub zapoznać się z opiniami użytkowników na forach. Dzięki analizie zdania użytkowników aplikacja stanie się bardziej intuicyjna.
Kolejnym, bardziej przyszłościowym kierunkiem byłaby synchronizacja z kontem Google lub Microsoft. Dzięki temu moglibyśmy rozszerzyć funkcjonalności o posiadanie kalendarza z wyświetlonymi datami z zadaniami jak i informacje odnośnie grafiku pracowniczego. Moglibyśmy również mieć odnośniki przy poszczególnych użytkownikach do ich e-mail przez co komunikacja stałaby się o wiele łatwiejsza.  Jest to oczywiście proces, w którym aplikacja utraci cechę prostej, zwartej aplikacji, ale niestety będzie zależała też od innych, zewnętrznych rozwiązań. Dany pomysł na rozbudowę aplikacji możemy kontynuować za pomocą rozbudowy informacji na temat użytkownika, przypięcie do aplikacji osobistego kanału smtp, aby użytkownicy dostawali przypisane do ich kont skrzynki pocztowe mogli komunikować się między sobą wewnętrznie za pomocą utworzonych służbowych kont. Kalendarz synchronizujący grafik pracowniczy z datami ukończenia poszczególnych zadań też jest funkcjonalnością, która może zostać wdrożona bez obecności zewnętrznych aplikacji.

