\chapter{Instalacja i wdrożenie}
\thispagestyle{chapterBeginStyle}
Rozdział ten przedstawia instrukcje dotyczące uruchomienia aplikacji.
\section{Uruchomienie serwera}
 Osoba zainteresowana, która chciałaby uruchomić daną aplikację lokalnie na swojej maszynie, musi liczyć się z tym, że rozbudowana aplikacja posiada wiele zależności z innymi programami i konfiguracji, dlatego serwer oraz baza danych zostały skonteneryzowane.  Konteneryzacja to wprowadzenie aplikacji ze wszystkimi zależnościami i wymaganymi do uruchomienia programami do wirtualnej jednostki zwanej kontenerem. Kontener to proces odizolowany od wszystkich innych procesów na maszynie hosta. Proces ten przeprowadzony został za pomocą programu \textit{Docker} \cite{docker}. 
 \paragraph{Instalacja programu Docker} 
 Platforma jest w pełni kompatybilna z systemem \textit{linux}, dla systemu \linebreak macOS i Windows docker udostępnia nakładkę pozwalającą korzystać z funkcji programu. Można pobrać go z oficjalnej strony :  \textit{\url{https://docs.docker.com/get-docker/}} 
 Po zainstalowaniu \textit{Dockera} w głównej gałęzi projektu \textit{./tmapp/}, gdzie znajduje się plik \textbf{docker-compose.yml}  wykonać w terminalu komendę:
 
 \begin{lstlisting}
docker-compose up --build -d
 \end{lstlisting}


Stworzone zostaną dwa kontenery, które przetrzymują serwer aplikacji i serwer bazy danych. Aby sprawdzić informacje na temat kontenerów należy wykonać komendę:
 
 \begin{lstlisting}
docker ps
\end{lstlisting}

Wystarczy pobrać \textit{dockera}, aby wywołać odpowiednie kontenery, które same doinstalują i zbudują projekt serwera. Aplikacja klienta nie została zdokeryzowana (inaczej konteneryzacja dzięki \textit{Docker}), ponieważ nie było to optymalne rozwiązanie. Framework \textit{Angular} wczytuje sporą liczbę plików przy budowaniu projektu, dlatego kontener ważyłby o wiele więcej. Podczas próby konteneryzacji trzech elementów, zauważony został zauważalny wzrost czasu załadowywania widoków na stronie oraz wydłużyła się komunikacja między serwerem.
\clearpage
\section{Uruchomienie klienta}
Kolejnym krokiem jest uruchomienie aplikacji klienta.
W tym celu potrzebujemy środowisko \textit{node.js}, aby zainstalować szkielet \textit{Angular}. W tym celu pobieramy ze strony: \textit{\url{https://nodejs.org/en/download/}} 
Podczas instalacji należy kierować się domyślnymi konfiguracjami, które dodają odpowiednio program \textit{node} oraz \textit{npm}. Po instalacji można sprawdzić czy dane programy zostały odpowiednio zainstalowane za pomocą komendy:
 \begin{lstlisting}
    node --version
\end{lstlisting}
oraz:
\begin{lstlisting}
    npm --version
\end{lstlisting}
Powinny zostać wyświetlone informacje na temat programu i jego zainstalowanej wersji. 
Kolejnym krokiem jest instalacja szkieletu \textit{Angular} za pomocą programów wcześniej zainstalowanych. 
Należy w wierszu poleceń wpisać następującą komendę:
\begin{lstlisting}
    npm install -g @angular/cli
\end{lstlisting}
Dzięki wszystkim programom zainstalowanym powyższymi komendami możemy ostatecznie przejść do głównego folderu z aplikacją w wierszy poleceń: \begin{lstlisting}
    cd /path/to/rootfolder/tmapp/
\end{lstlisting}
Kiedy wiersz polecenia wskazuje na główny folder systemu, w którym znajduje się plik \textit{package.json}, zbudujemy projekt za pomocą komendy:

\begin{lstlisting}
    npm install
\end{lstlisting}

Po pomyślnej instalacji wszystkich zależności potrzebnych do uruchomienia projektu powinien zostać wyświetlony komunikat o sukcesie.
Ostatnim krokiem zostało włączenie aplikacji za pomocą komendy:
\begin{lstlisting}
    ng serve
\end{lstlisting}
Gdy dany proces włączania aplikacji przejdzie pomyślnie, strona \textit{http://localhost:9000} zostanie włączona automatycznie w domyślnej przeglądarce internetowej danego systemu.





