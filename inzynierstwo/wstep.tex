\chapter{Wstęp}
\thispagestyle{chapterBeginStyle}

\section{Wprowadzenie}
Optymalizacja procesów, kosztów czy czasu pracy to kluczowe zagadnienia, z którymi borykają się zarówno duże, jak i małe przedsiębiorstwa. Problem dotyczy również pojedynczych osób, które chcą wykorzystać produktywnie otrzymaną dobę, nie zaniedbując żadnej sfery życia. Przedsiębiorcy zarządzają kosztami, jakością produktu oraz czasem pracowników, tak aby uzyskać zadowolenie klienta przy jednoczesnym zysku dla firmy. Konstruktorzy przy zachowaniu norm i przepisów starają się znaleźć optymalne proporcję pomiędzy zachowaniem odpowiednich parametrów konstrukcyjnych, a kosztami budowy. Tak więc optymalizacja może być rozumiana jako poszukiwanie najlepszego rozwiązania na podstawie wybranego kryterium. Kierując się chęcią optymalizacji i wzrostu produktywności pracy przy zastosowaniu prostych technik, powstał pomysł na aplikację wspomagającą zarządzaniem zadaniami. Obecnie znajomość podejścia takiego jak Kanban, czy Agile w dziedzinach związanych z informatyką jest oczekiwana przez pracodawców. Wielkie firmy, w których pracuje rzesze pracowników korzystają z podanych podejść na co dzień. Za pomocą wielopoziomowych aplikacji, które posiadają skomplikowane funkcje, zostały zaimplementowane podstawowe podejścia Kanban lub Agile. Celem niniejszej pracy inżynierskiej jest próba spopularyzowania podejścia Kanban wśród mniejszych firmy, które nie są związane w żadnym stopniu z dziedziną informatyki. 
\section{Zakres pracy}
Zakres pracy obejmuje opisanie i stworzenie aplikacji wspierającej zarządzanie zadaniami w zespole pracującym nad dowolnym projektem przy pomocy technologii Java ze szkieletem architektonicznym Spring oraz Angular. Tematem niniejszej pracy jest aplikacja internetowa, która służy do komunikacji, archiwizowania i raportowania wykonywanych zadań wraz z ich przejrzystym wyświetlaniem. Przemyślana organizacja czasu i zarządzenie zadaniami zwiększa efektywność i pozwala na realizację długookresowych celów wymagających znacznych nakładów pracy lub kosztów. Zastosowanie aplikacji w przedsiębiorstwach pozwoli na zwiększenie obrotów, a tym samym możliwość na większe zyski.

 \section{Podział pracy}
Niniejsza praca inżynierska składa się z siedmiu rozdziałów. Rozdział pierwszy stanowi wprowadzenie do tematu, w którym wyjaśniono cel, zakres i podział pracy. W drugim rozdziale omówiono problematykę zarządzania zadaniami, proponowane rozwiązanie - podejście Kanban, porównano również istniejące rozwiązania na rynku aplikacji biznesowych. Zostały przeanalizowane dwie aplikacje Trello i LeanKit. Na zakończenie rozdziału zdefiniowano grupę potencjalnych użytkowników, chętnych do korzystania z aplikacji. W rozdziale trzecim przedstawiono projekt systemu, w postaci wymagań funkcjonalnych i niefunkcjonalnych oraz przypadków użycia. Cała aplikacja została zobrazowana za pomocą diagramu pakietów. Ostatecznie został zaprezentowany projekt bazy danych. W rozdziale czwartym został opisany wizualny aspekt aplikacji, czyli interfejs użytkownika. Przedstawione zostały najważniejsze panele aplikacji. W piątym rozdziale przedstawiono sposób implementacji, wykorzystywaną technologię, problemy implementacyjne, jak i opisy kodów źródłowych ważniejszych elementów aplikacji. W rozdziale szóstym przedstawiono sposób instalacji i wdrożenia systemu w środowisku docelowym. Ostatni rozdział jest podsumowaniem pracy nad aplikacją, jak i przedstawieniem przyszłościowych planów rozwoju aplikacji. 

